%2multibyte Version: 5.50.0.2960 CodePage: 1252

\documentclass{article}
%%%%%%%%%%%%%%%%%%%%%%%%%%%%%%%%%%%%%%%%%%%%%%%%%%%%%%%%%%%%%%%%%%%%%%%%%%%%%%%%%%%%%%%%%%%%%%%%%%%%%%%%%%%%%%%%%%%%%%%%%%%%%%%%%%%%%%%%%%%%%%%%%%%%%%%%%%%%%%%%%%%%%%%%%%%%%%%%%%%%%%%%%%%%%%%%%%%%%%%%%%%%%%%%%%%%%%%%%%%%%%%%%%%%%%%%%%%%%%%%%%%%%%%%%%%%
\usepackage{amsmath}

\setcounter{MaxMatrixCols}{10}
%TCIDATA{OutputFilter=LATEX.DLL}
%TCIDATA{Version=5.50.0.2960}
%TCIDATA{Codepage=1252}
%TCIDATA{<META NAME="SaveForMode" CONTENT="1">}
%TCIDATA{BibliographyScheme=Manual}
%TCIDATA{Created=Tuesday, June 13, 2017 09:19:44}
%TCIDATA{LastRevised=Tuesday, May 15, 2018 15:35:11}
%TCIDATA{<META NAME="GraphicsSave" CONTENT="32">}
%TCIDATA{<META NAME="DocumentShell" CONTENT="Standard LaTeX\Blank - Standard LaTeX Article">}
%TCIDATA{Language=American English}
%TCIDATA{CSTFile=40 LaTeX article.cst}

\newtheorem{theorem}{Theorem}
\newtheorem{acknowledgement}[theorem]{Acknowledgement}
\newtheorem{algorithm}[theorem]{Algorithm}
\newtheorem{axiom}[theorem]{Axiom}
\newtheorem{case}[theorem]{Case}
\newtheorem{claim}[theorem]{Claim}
\newtheorem{conclusion}[theorem]{Conclusion}
\newtheorem{condition}[theorem]{Condition}
\newtheorem{conjecture}[theorem]{Conjecture}
\newtheorem{corollary}[theorem]{Corollary}
\newtheorem{criterion}[theorem]{Criterion}
\newtheorem{definition}[theorem]{Definition}
\newtheorem{example}[theorem]{Example}
\newtheorem{exercise}[theorem]{Exercise}
\newtheorem{lemma}[theorem]{Lemma}
\newtheorem{notation}[theorem]{Notation}
\newtheorem{problem}[theorem]{Problem}
\newtheorem{proposition}[theorem]{Proposition}
\newtheorem{remark}[theorem]{Remark}
\newtheorem{solution}[theorem]{Solution}
\newtheorem{summary}[theorem]{Summary}
\newenvironment{proof}[1][Proof]{\noindent\textbf{#1.} }{\ \rule{0.5em}{0.5em}}
\input{tcilatex}
\begin{document}


\begin{center}
\textbf{Computer Tutorial 1: Bayesian VARs}
\end{center}

In this exercise sheet, I\ provide a series of questions and answers
relating to two Bayesian VARs. Given that I\ am providing the answers, what
do I expect you to do in the computer sessions?\ First I\ am not expecting
you to go through the theoretical derivations and proofs (parts a and b of
Exercise 1 and part a)\ of Exercise 2). However, since I\ do not have time
to do proofs in the lectures, I\ felt I\ should make them available to you
in case you want to see them or have them as a resource for your future
Bayesian research. So I\ suggest you do only a quick reading of this
exercise sheet (without worrying about the details of the proofs) before the
computer tutorial to get an overview of the model each exercise relates to.
Instead focus on the computational part of the each exercise (parts c)\ and
b), respectively, of the two exercises). Code which does these exercises is
provided. In the computer lab, I suggest you experiment with these codes
(e.g. try different priors, lag lengths, etc.)\ to familiarize yourself with
Bayesian programming in VARs.

Some of the notation I\ use in this exercise sheet is different than that
used in the lecture.

\newpage

\begin{center}
\textbf{Exercise 1 (The Vector Autoregressive Model with Independent
Normal-Wishart Prior)}
\end{center}

Suppose $y_{t}=(y_{1t},\ldots ,y_{nt})^{\prime }$ is a vector of $n$
dependent variables at time $t$ for $t=1,\ldots ,T$. Then the VAR($p$) is
defined as: 
\begin{equation}
y_{t}=b+A_{1}y_{t-1}+\cdots +A_{p}y_{t-p}+\epsilon _{t},  \label{VAR-y}
\end{equation}%
where $b$ is an $n\times 1$ vector of intercepts, $A_{1},\ldots ,A_{p}$ are $%
n\times n$ coefficient matrices and $\epsilon _{t}\sim N(0,\Sigma )$. In
other words, the VAR($p$) is simply a multiple-equation regression where the
regressors are the lagged dependent variables.\footnote{%
Note that the VAR depends on $p$ initial conditions $y_{-p+1},\ldots ,y_{0}$%
. In principle these initial conditions can be modeled explicitly. However,
it is common to treat them as fixed (i.e. the first $p$ observations used as
initial conditions and the estimation sample begins with the $\left(
p+1\right) ^{th}$ observation). We adopt this strategy in these exercises.}

The VAR($p$) can be written as: 
\begin{equation*}
y_{t}=X_{t}\beta +\epsilon _{t},
\end{equation*}%
where $X_{t}=I_{n}\otimes \lbrack 1,y_{t-1}^{\prime },\ldots
,y_{t-p}^{\prime }]$ and $\beta =\text{vec}([b,\,A_{1},\,\cdots
\,A_{p}]^{\prime }).$ Alternatively, the observations can be stacked over $%
t=1,\ldots ,T$ \ so that the VAR($p$) can be written as 
\begin{equation*}
y=X\beta +\epsilon
\end{equation*}%
where $\epsilon \sim N(0,I_{T}\otimes \Sigma )$.

(a) Derive the likelihood function for the VAR.

(b) Assume that the priors for $\beta $ and $\Sigma $ are independent of one
another and take the form: 
\begin{equation}
\beta \sim N(\beta _{0},V_{\boldsymbol{\beta }}),\quad \Sigma \sim IW(\nu
_{0},S_{0})  \label{VAR-prior}
\end{equation}%
where $IW(.,.)$ is the inverse-Wishart distribution. This is often referred
to as the independent Normal-Wishart prior. Derive a Gibbs sampler which
allows for posterior inference in the VAR.

(c)\ Use the data set US\_macrodata.csv provided which contains US quarterly
data on the CPI inflation rate, the unemployment rate and Fed Funds rate
from 1959Q2 to 2007Q4---the sample ends at 2007Q4 to avoid the periods when
interest rate hits the zero lower bound. Estimate a VAR\ using this data set
and the Gibbs sampler derived in part b). Calculate impulse responses of all
variables to a monetary policy shock.

\textbf{Solution to Exercise 1}

(a) Since the VAR($p$) is a multivariate regression model, standard Bayesian
derivations used with multivariate Normal analysis apply (see Exercise
2.14). In particular, since 
\begin{equation*}
(y\,|\,\beta ,\Sigma )\sim N(X\beta ,I_{T}\otimes \Sigma ),
\end{equation*}%
the likelihood function is given by: 
\begin{align}
p(y\,|\,\beta ,\Sigma )& =|2\pi (I_{T}\otimes \Sigma )|^{-\frac{1}{2}}\text{e%
}^{-\frac{1}{2}(y-X\beta )^{\prime }(I_{T}\otimes \Sigma )^{-1}(y-X\beta )} 
\notag \\
& =(2\pi )^{-\frac{Tn}{2}}|\Sigma |^{-\frac{T}{2}}\text{e}^{-\frac{1}{2}%
(y-X\beta )^{\prime }(I_{T}\otimes \Sigma ^{-1})(y-X\beta )},
\label{VAR-like1}
\end{align}%
where the second equality holds because $|I_{T}\otimes \Sigma |=|\Sigma
|^{T} $ and $(I_{T}\otimes \Sigma )^{-1}=I_{T}\otimes \Sigma ^{-1}$. Note
that the likelihood can also be written as 
\begin{equation}
p(y\,|\,\beta ,\Sigma )=(2\pi )^{-\frac{Tn}{2}}|\Sigma |^{-\frac{T}{2}}\text{%
e}^{-\frac{1}{2}\sum_{t=1}^{T}(y_{t}-X_{t}\beta )^{\prime }\Sigma
^{-1}(y_{t}-X_{t}\beta )}.  \label{VAR-like2}
\end{equation}

(b) We derive a Gibbs sampler for the VAR($p$) with likelihood given in (\ref%
{VAR-like1}) and priors given in (\ref{VAR-prior}). Specifically, we derive
the two conditional densities $p(\beta \,|\,y,\Sigma )$ and $p(\Sigma
\,|\,y,\beta )$.

The first step is straightforward, as standard linear regression results
apply (see Chapter 10 of Bayesian Econometric Methods). In fact, we have 
\begin{equation*}
(\beta \,|\,y,\Sigma )\sim N(\hat{\beta},K_{\beta }^{-1}),
\end{equation*}%
where 
\begin{equation*}
K_{\beta }=V_{\beta }^{-1}+X^{\prime }(I_{T}\otimes \Sigma ^{-1})X,\quad 
\hat{\beta}=K_{\beta }^{-1}\left( V_{\beta }^{-1}\beta _{0}+X^{\prime
}(I_{T}\otimes \Sigma ^{-1})y\right) ,
\end{equation*}%
and we have used the result $(I_{T}\otimes \Sigma )^{-1}=I_{T}\otimes \Sigma
^{-1}$.

Next, we derive the conditional density $p(\Sigma \,|\,y,\beta )$. We will
use the fact that, for conformable matrices $A,B,C$, we have 
\begin{equation*}
\text{tr}(ABC)=\text{tr}(BCA)=\text{tr}(CAB).
\end{equation*}%
Combining the likelihood in (\ref{VAR-like1}) with the inverse-Wishart prior
for $\Sigma $, we obtain 
\begin{align*}
p(\Sigma \,|\,y,\beta )& \propto p(y\,|\,\beta ,\Sigma )p(\Sigma ) \\
& \propto |\Sigma |^{-\frac{T}{2}}\text{e}^{-\frac{1}{2}%
\sum_{t=1}^{T}(y_{t}-X_{t}\beta )^{\prime }\Sigma ^{-1}(y_{t}-X_{t}\beta
)}\times |\Sigma |^{-\frac{\nu _{0}+n+1}{2}}\text{e}^{-\frac{1}{2}\text{tr}%
(S_{0}\Sigma ^{-1})} \\
& \propto |\Sigma |^{-\frac{\nu _{0}+n+T+1}{2}}\text{e}^{-\frac{1}{2}\text{tr%
}(S_{0}\Sigma ^{-1})}\text{e}^{-\frac{1}{2}\text{tr}\left[
\sum_{t=1}^{T}(y_{t}-X_{t}\beta )(y_{t}-X_{t}\beta )^{\prime }\Sigma ^{-1}%
\right] } \\
& \propto |\Sigma |^{-\frac{\nu _{0}+n+T+1}{2}}\text{e}^{-\frac{1}{2}\text{tr%
}\left[ \left( S_{0}+\sum_{t=1}^{T}(y_{t}-X_{t}\beta )(y_{t}-X_{t}\beta
)^{\prime }\right) \boldsymbol{\Sigma }^{-1}\right] },
\end{align*}%
which is the kernel of an inverse-Wishart density (type \textquotedblleft
inverse-Wishart\textquotedblright\ into Wikipedia to see this). Thus, we
have 
\begin{equation*}
(\Sigma \,|\,y,\beta )\sim IW\left( \nu
_{0}+T,S_{0}+\sum_{t=1}^{T}(y_{t}-X_{t}\beta )(y_{t}-X_{t}\beta )^{\prime
}\right) .
\end{equation*}

We summarize the Gibbs sampler as follows

\begin{algorithm}
\textrm{\textbf{(Gibbs Sampler for the VAR($p$)).} }

\textrm{Pick some initial values }$\mathrm{\beta }^{\mathrm{(0)}}\mathrm{=c}%
_{\mathrm{0}}$\textrm{\ and }$\mathrm{\Sigma }^{\mathrm{(0)}}\mathrm{=C}_{%
\mathrm{0}}\mathrm{>0}$\textrm{. Then, repeat the following steps from }$%
\mathrm{r=1}$\textrm{\ to }$\mathrm{R}$\textrm{: }

\begin{enumerate}
\item \textrm{\ Draw }$\mathrm{\beta }$\textrm{$^{(r)}\sim p(\beta
\,|\,y,\Sigma ^{(r-1)})$ (multivariate Normal). }

\item \textrm{Draw }$\mathrm{\Sigma }^{\mathrm{(r)}}$\textrm{$\sim p(\Sigma
\,|\,y,\beta ^{(r)})$ (inverse-Wishart). }
\end{enumerate}
\end{algorithm}

(c) Matlab code which answers this question and produces the figures below
is provided. Empirical results below use a VAR(2) and sets $\beta _{0}=0,V_{%
\boldsymbol{\beta }}=I$ (except for the diagonal elements of $V_{\boldsymbol{%
\beta }}$ which correspond to the intercepts in each equation which are set
to $10$), $\nu _{0}=6,S_{0}=I$. In order to identify the monetary policy
shocks we order the interest rate last and treat it as the monetary policy
instrument.

We first implement the Gibbs sampler described in part b). Then, given the
posterior draws of $\beta $ and $\Sigma $, we compute the impulse-response
functions of the three variables to a 100-basis-point monetary policy shock.
Thus, the MATLAB code contains several scripts. The main script is VAR.m. It
loads the dataset, constructs the regression matrix $X$ using the script
SURform2.m.

For each of the posterior draws of $\beta $ and $\Sigma $, we use the script
construct\_IR.m to compute the impulse-response functions of the three
variables to a 100-basis-point monetary policy shock. More specifically, we
consider two alternative paths: in one a 100-basis-point monetary policy
shock hits the system, and in the other this shock is absent. We then let
the two systems evolve according to the VAR($p$) written as the regression 
\begin{equation*}
y_{t}=X_{t}\beta _{t}+C\tilde{\epsilon}_{t},\quad \tilde{\epsilon}_{t}\sim
N(0,I_{3}),
\end{equation*}%
for $t=1,\ldots ,n_{\mathrm{hz}}$, where $n_{\mathrm{hz}}$ denotes the
number of horizons and $C$ is the Cholesky factor of $\Sigma $. Each
impulse-response function is then the difference between these two paths.

The impulse-response functions of inflation, unemployment and interest rate
to a 100-basis-point monetary policy shock are given in the figure below.

\FRAME{itbpFU}{5.604in}{2.1136in}{0in}{\Qcb{Impulse-response functions of
inflation (left panel), unemployment (middle panel) and interest rate (right
panel) to a 100-basis-point monetary policy shock.}}{}{var_ir.eps}{\special%
{language "Scientific Word";type "GRAPHIC";maintain-aspect-ratio
TRUE;display "USEDEF";valid_file "F";width 5.604in;height 2.1136in;depth
0in;original-width 8.323in;original-height 3.1125in;cropleft "0";croptop
"1";cropright "1";cropbottom "0";filename 'VAR_IR.eps';file-properties
"XNPEU";}}

By construction, the interest rate increases by 1\% on impact, and it
gradually goes back to zero over the next 20 quarters. The impact on
unemployment is delayed and positive---the unemployment rate rises slowly
and reaches a peak of about 0.25\% after 3 years on impact.

In contrast, the inflation rate rises shortly after impact, and the effect
is positive. The literature often refers to this empirical finding as the
price puzzle---a contractionary monetary policy shock should dampen rather
than increase inflation. Some argue that this reflects
misspecification---some crucial variables are omitted, and both interest
rate and inflation are responding to these omitted variables. Alternatively,
Primiceri (2005) argues that the monetary policy shocks identified in this
context should be interpreted as non-systematic policy actions that capture
both policy mistakes and interest rate movements that are responses to
variables other than inflation and unemployment.

\newpage

\textbf{Exercise 2 (The VAR\ with Minnesota Prior)}

For this exercise, we use the notation and data set of Exercise 1. The
Minnesota prior has traditionally been the most popular prior for Bayesian
VARs. It involves a simplification in that $\Sigma $ is assumed to be a
diagonal matrix and replaced with an estimate, $\widehat{\Sigma }$, where $%
\widehat{\sigma }_{ii}=s_{i}^{2}$ (where $s_{i}^{2}$ is the standard OLS
estimate of the error variance using an AR model for the $i^{th}$ variable
and $\widehat{\sigma }_{ii}$ is the $ii^{th}$ element of $\widehat{\Sigma }$%
). To explain the prior for the vector of VAR($p$)\ coefficients, $\beta $,
note that some elements correspond to own lags (e.g. in the equation for the
first variable in the VAR\ there are lags of the first variable), some
elements correspond to other lags (e.g. in the equation for the first
variable in the VAR\ there are lags of the second and third variables) and
some to exogenous or deterministic variables (e.g. the intercept). The
Minnesota prior reflects a belief that own lags are more likely to be
important explanatory variables than other lags and more recent lags are
likely to be more important than those in the more distant past.

The Minnesota prior assumes $\beta \sim N(\beta _{0},V_{\boldsymbol{\beta }%
}) $. Consider first, the prior mean, $\beta _{0}$. When using stationary
variables which exhibit little persistence (e.g. the growth rates of
macroeconomic variables) it is common to simply set $\beta _{0}=0$. When
working with nonstationary data (e.g. the logs of macroeconomic variables)
it is common to set $\beta _{0}=0$ except for the elements corresponding to
the first own lag of the dependent variable in each equation. These elements
are set to one. Next consider the prior covariance matrix, $V_{\boldsymbol{%
\beta }}$, which is assumed to be a diagonal matrix. The Minnesota prior
sets the diagonal elements of $V_{\boldsymbol{\beta }}$ as follows:

\begin{itemize}
\item $\frac{\underline{a}_{1}}{r^{2}}$ for coefficients on own lag $r$ for $%
r=1,..,p$

\item $\frac{\underline{a}_{2}\widehat{\sigma }_{ii}}{r^{2}\widehat{\sigma }%
_{jj}}$ for coefficients on lag $r$ of variable $j\neq i$ for $r=1,..,p$ in
equation $i$

\item $\underline{a}_{3}\widehat{\sigma }_{ii}$ for coefficients on
deterministic or exogenous variables in equation $i$.
\end{itemize}

The Minnesota prior thus turns a complicated prior elicitation procedure
into a much simpler one which requires the research to select three scalar
prior hyperparameters, $\underline{a}_{1},\underline{a}_{2}$ and $\underline{%
a}_{3}$. Typically the researcher sets $\underline{a}_{1}>\underline{a}_{2}.$
In cases where the number of deterministic terms is small (e.g. if the only
such terms are intercepts) then $\underline{a}_{3}$ is set to a large
number, reflecting relatively non-informative prior beliefs.

(a) Derive the posterior for the VAR\ parameters using the Minnesota prior.
Discuss what computational methods can be used to carry out Bayesian
inference with this posterior.

(b) Use the data set described in Exercise 1. Estimate a VAR\ with the
Minnesota prior using this data set.

\textbf{Solution to Exercise 2}

(a) This derivation involves working out the posterior for $\beta $
conditional on $\Sigma =\widehat{\Sigma }$. But an identical derivation was
done in Exercise 1 as part of the derivation of the Gibbs sampler in that
exercise. Thus, we can simply use our previous derivation to say: 
\begin{equation*}
(\beta \,|\,y)\sim N(\hat{\beta},K_{\beta }^{-1}),
\end{equation*}%
where 
\begin{equation*}
K_{\beta }=V_{\beta }^{-1}+X^{\prime }(I_{T}\otimes \widehat{\Sigma }%
^{-1})X,\quad \hat{\beta}=K_{\beta }^{-1}\left( V_{\beta }^{-1}\beta
_{0}+X^{\prime }(I_{T}\otimes \widehat{\Sigma }^{-1})y\right) ,
\end{equation*}%
where the Minnesota prior forms for $\beta _{0}$ and $V_{\beta }$ are used.
This implies that posterior inference can be done analytically. That is, the
posterior mean of the VAR\ coefficients is $\hat{\beta}$ and the posterior
covariance matrix is $K_{\beta }^{-1}$. The properties of the Normal
distribution can be used to calculate other features such as credible
intervals.

Thus, there is no need to use any posterior simulation method with the
Minnesota prior unless interest centers on nonlinear functions of $\beta $
such as impulse response functions. Monte Carlo integration can be used to
carry out posterior inference on features such as impulse response
functions. That is, draws can be taken from the Normal posterior for $\beta $
and each transformed to produce an impulse response function. Averages of
these draws will converge to the posterior mean of the impulse response
function.

(b) Matlab code (named bvar\_minnesota.m) which answers this question is
provided. The results below set $p=1$ and $\underline{a}_{1}=0.5,\underline{a%
}_{2}=0.25$ and $\underline{a}_{3}=100$ and include an intercept. We have
also set $\beta _{0}=0$ (although the user may wish to experiment with other
values as the variables do display a fairly high degree of persistence). The
following table presents estimates of the VAR\ coefficients along with their
posterior standard deviations. Each column of the table presents results for
one of the three equations in the VAR. If we use the rule of thumb which
says a variable has important explanatory power if its posterior mean is
more than two posterior standard deviations from zero, then the first own
lag is important in every equation. Beyond this, it is only the unemployment
equation which has any important explanatory variables.

\begin{center}
\begin{tabular}{|l|l|l|l|}
\hline
\multicolumn{4}{|l|}{%
\begin{tabular}{l}
Table 21.1:\ Posterior Results using Minnesota Prior for the VAR \\ 
Posterior mean (stand. dev. in parentheses)%
\end{tabular}%
} \\ \hline
& Inflation & Unemployment & Interest Rate \\ \hline
Lag inflation & 
\begin{tabular}{l}
$0.701$ \\ 
$\left( 0.057\right) $%
\end{tabular}
& 
\begin{tabular}{l}
$0.088$ \\ 
$\left( 0.040\right) $%
\end{tabular}
& 
\begin{tabular}{l}
$0.212$ \\ 
$\left( 0.128\right) $%
\end{tabular}
\\ \hline
Lag unemployment & 
\begin{tabular}{l}
$-0.028$ \\ 
$\left( 0.022\right) $%
\end{tabular}
& 
\begin{tabular}{l}
$0.953$ \\ 
$\left( 0.016\right) $%
\end{tabular}
& 
\begin{tabular}{l}
$-0.050$ \\ 
$\left( 0.050\right) $%
\end{tabular}
\\ \hline
Lag interest rate & 
\begin{tabular}{l}
$0.038$ \\ 
$\left( 0.013\right) $%
\end{tabular}
& 
\begin{tabular}{l}
$0.023$ \\ 
$\left( 0.009\right) $%
\end{tabular}
& 
\begin{tabular}{l}
$0.927$ \\ 
$\left( 0.303\right) $%
\end{tabular}
\\ \hline
\end{tabular}
\end{center}

\end{document}
